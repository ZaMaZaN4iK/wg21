\documentclass{wg21}

\usepackage{xcolor}
\usepackage{soul}
\usepackage{ulem}
\usepackage{fullpage}
\usepackage{parskip}
\usepackage{csquotes}
\usepackage{listings}
\usepackage{minted}
\usepackage{enumitem}

\lstdefinestyle{base}{
  language=c++,
  breaklines=false,
  basicstyle=\ttfamily\color{black},
  moredelim=**[is][\color{green!50!black}]{@}{@},
  escapeinside={(*@}{@*)}
}

\newcommand{\cc}[1]{\mintinline{c++}{#1}}
\newminted[cpp]{c++}{}


\title{Add \cc{std::is_partitioned_until} algorithm}
\docnumber{P1927R1}
\audience{LEWGI}
\author{Alexander Zaitsev}{zamazan4ik@tut.by, zamazan4ik@gmail.com}

\begin{document}
\maketitle

\section{Revision history}
\begin{itemize}
  \item R1
  \begin{itemize}
  	\item Add a remark about naming: why \cc{is_*} naming was chosen
  	\item Change for the parallel version signature: use ForwardIterator instead of InputIterator (for consistency with the existing parallel algorithms and also likely a prerequisite for implementers to be able to implement this)
  	\item Change return value for the third overload: from \cc{input_iterator} to \cc{I}
  	\item Change return value for the forth overload: from \cc{iterator_t<R>} to \cc{safe_iterator_t<R>}
  \end{itemize}
  \item R0
  \begin{itemize}
  	\item Initial draft
  \end{itemize}
\end{itemize}

\section{Motivation}
\cc{std::is_partitioned} was added long time ago to the standard library. The algorithm is useful but sometimes we need an infromation about "where a partition is broken". \cc{std::is_partitioned} returns only \cc{bool} and we cannot change an interface of existing function. So we can add additional function \cc{std::is_partitioned_until} which returns an iterator instead of bool. \cc{partition_point} cannot be used here because it works only on partitioned ranges.

\section{Naming}
\cc{std::is_partitioned_until} as a name was chosen only for being consistent with already defined in the standard library \cc{is_sorted_until} function. If anyone can propose better name - it can be discussed again (however I didn't find better name for such function). But from my point of view we shall be consistent here with existent functionality in the standard library.

\section{Proposed wording}
Add to \textbf{[alg.partitions] 25.7.4}:
\begin{quote}
[...]	
\begin{itemdecl}
template<class InputIterator, class Predicate>
  constexpr InputIterator is_partitioned_until(InputIterator first, InputIterator last,
                                               Predicate pred);
template<class ExecutionPolicy, class InputIterator, class Predicate>
  ForwardIterator is_partitioned_until(ExecutionPolicy&& exec, ForwardIterator first,
                                       ForwardIterator last, Predicate pred);
\end{itemdecl}

\begin{itemdecl}
template<input_iterator I, sentinel_for<I> S, class Proj = identity,
         indirect_unary_predicate<projected<I, Proj>> Pred>
  constexpr I ranges::is_partitioned_until(I first, S last, Pred pred,
                                           Proj proj = {});	
template<input_range R, class Proj = identity, 
         indirect_unary_predicate<projected<iterator_t<R>, Proj>> Pred>
  constexpr borrowed_iterator_t<R> ranges::is_partitioned_until(R&& r, Pred pred,
                                                                Proj proj = {});
\end{itemdecl}
    Let proj be identity{} for the overloads with no parameter named proj.
    
    Returns: the last iterator \cc{it} in the sequence [first, last) for which the \cc{is_partitioned(first, it)} is true.
    
    Complexity: Linear. At most last - first applications of pred and proj.
[...]
\end{quote}

\section{Examples}
Given the container c containing \cc{ 0,1,2,3,14,15 }, then
\begin{quote}
\begin{itemdecl}    
bool isOdd ( int i ) { return i % 2 == 1; }
bool lessThan10 ( int i ) { return i < 10; }
    
is_partitioned_until ( c, isOdd ) // iterator to '1'
is_partitioned_until ( c, lessThan10 ) // end
is_partitioned_until ( c.begin (), c.end (), lessThan10 ) // end
is_partitioned_until ( c.begin (), c.begin () + 3, lessThan10 ) // end
is_partitioned_until ( c.end (), c.end (), isOdd ) // end, because of empty range
\end{itemdecl}
\end{quote}

\section{Possible implementation}
Also possible implementation can be found in Boost.Algorithm: \href{https://github.com/boostorg/algorithm/blob/master/include/boost/algorithm/is_partitioned_until.hpp}{GitHub}. Documentation can be found here: \href{https://www.boost.org/doc/libs/1_70_0/libs/algorithm/doc/html/the_boost_algorithm_library/Misc/is_partitioned_until.html}{Boost}. Available in Boost.Algorithm since Boost 1.65.

\end{document}
