\documentclass{wg21}

\usepackage{xcolor}
\usepackage{soul}
\usepackage{ulem}
\usepackage{fullpage}
\usepackage{parskip}
\usepackage{csquotes}
\usepackage{listings}
\usepackage{minted}
\usepackage{enumitem}

\lstdefinestyle{base}{
  language=c++,
  breaklines=false,
  basicstyle=\ttfamily\color{black},
  moredelim=**[is][\color{green!50!black}]{@}{@},
  escapeinside={(*@}{@*)}
}

\newcommand{\cc}[1]{\mintinline{c++}{#1}}
\newminted[cpp]{c++}{}


\title{Making \cc{std::list} constexpr}
\docnumber{P1922R0}
\audience{LEWGI}
\author{Alexander Zaitsev}{zamazan4ik@tut.by, zamazan4ik@gmail.com}

\begin{document}
\maketitle

\section{Revision history}
\begin{itemize}
  \item R0 -- Initial draft
\end{itemize}


\section{Abstract}
\cc{std::list} is not currently \cc{constexpr} friendly. With the loosening
of requirements on \cc{constexpr} in \cite{P0784R1} and related papers, we
can now make \cc{std::list} \cc{constexpr}, and we should in order to support
the \cc{constexpr} reflection effort (and other evident use cases).

\section{Motivation}
\cc{std::list} is not so widely-used standard container as \cc{std::vector} or \cc{std::string}. But there is no reason to keep \cc{std::list} in non-\cc{constexpr} state since one of the main directions of C++ evolution is compile-time programming. And we want to use in compile-time as much as possible from STL. And this paper makes \cc{std::list} available in compile-time.

\section{Proposed wording}
We basically mark all the member and non-member functions of \cc{std::list} \cc{constexpr}.

Direction to the editor: please apply \cc{constexpr} to all of \cc{std::list},
including any additions that might be missing from this paper.

In \textbf{[support.limits.general]}, add the new feature test macro
\cc{__cpp_lib_constexpr_list} with the corresponding value for header
\cc{<list>} to Table 36 \textbf{[tab:support.ft]}.

Change in \textbf{[list.syn] 22.3.5}:
\begin{quote}
\begin{codeblock}
#include <initializer_list>

namespace std {
  // 22.3.10, class template \tcode{list}
  template<class T, class Allocator = allocator<T>> class list;

  template<class T, class Allocator>
    @\added{constexpr}@ bool operator==(const list<T, Allocator>& x, const list<T, Allocator>& y);
  template<class T, class Allocator>
    @\added{constexpr}@ synth-three-way-result<T> operator<=>(const list<T, Allocator>& x, const list<T, Allocator>& y);

  template<class T, class Allocator>
    @\added{constexpr}@ void swap(list<T, Allocator>& x, list<T, Allocator>& y)
      noexcept(noexcept(x.swap(y)));
      
  template<class T, class Allocator, class U>
    @\added{constexpr}@ void erase(list<T, Allocator>& c, const U& value);
  template<class T, class Allocator, class Predicate>
    @\added{constexpr}@ void erase_if(list<T, Allocator>& c, Predicate pred);    

  [...]
}
\end{codeblock}
\end{quote}

Add after \textbf{[list.overview] 22.3.10.1/2}:
\begin{quote}
\added{The types \texttt{iterator} and \texttt{const_iterator} meet the
constexpr iterator requirements ([iterator.requirements.general]).}
\end{quote}

Change in \textbf{[list.overview] 22.3.10.1}:
\begin{quote}
\begin{codeblock}
namespace std {
  template<class T, class Allocator = allocator<T>>
  class list {
  public:
    // types
    using value_type             = T;
    using allocator_type         = Allocator;
    using pointer                = typename allocator_traits<Allocator>::pointer;
    using const_pointer          = typename allocator_traits<Allocator>::const_pointer;
    using reference              = value_type&;
    using const_reference        = const value_type&;
    using size_type              = @\impdef@; // see 22.2
    using difference_type        = @\impdef@; // see 22.2
    using iterator               = @\impdef@; // see 22.2
    using const_iterator         = @\impdef@; // see 22.2
    using reverse_iterator       = std::reverse_iterator<iterator>;
    using const_reverse_iterator = std::reverse_iterator<const_iterator>;

    // 22.3.10.2, construct/copy/destroy
    @\added{constexpr}@ list() : list(Allocator()) { }
    @\added{constexpr}@ explicit list(const Allocator&);
    @\added{constexpr}@ explicit list(size_type n, const Allocator& = Allocator());
    @\added{constexpr}@ list(size_type n, const T& value, const Allocator& = Allocator());
    template<class InputIterator>
      @\added{constexpr}@ list(InputIterator first, InputIterator last, const Allocator& = Allocator());
    @\added{constexpr}@ list(const list& x);
    @\added{constexpr}@ list(list&&);
    @\added{constexpr}@ list(const list&, const Allocator&);
    @\added{constexpr}@ list(list&&, const Allocator&);
    @\added{constexpr}@ list(initializer_list<T>, const Allocator& = Allocator());
    @\added{constexpr}@ ~list();
    @\added{constexpr}@ list& operator=(const list& x);
    @\added{constexpr}@ list& operator=(list&& x)
      noexcept(allocator_traits<Allocator>::is_always_equal::value);
    @\added{constexpr}@ list& operator=(initializer_list<T>);
    template<class InputIterator>
      @\added{constexpr}@ void assign(InputIterator first, InputIterator last);
    @\added{constexpr}@ void assign(size_type n, const T& u);
    @\added{constexpr}@ void assign(initializer_list<T>);
    @\added{constexpr}@ allocator_type get_allocator() const noexcept;

    // iterators
    @\added{constexpr}@ iterator               begin() noexcept;
    @\added{constexpr}@ const_iterator         begin() const noexcept;
    @\added{constexpr}@ iterator               end() noexcept;
    @\added{constexpr}@ const_iterator         end() const noexcept;
    @\added{constexpr}@ reverse_iterator       rbegin() noexcept;
    @\added{constexpr}@ const_reverse_iterator rbegin() const noexcept;
    @\added{constexpr}@ reverse_iterator       rend() noexcept;
    @\added{constexpr}@ const_reverse_iterator rend() const noexcept;

    @\added{constexpr}@ const_iterator         cbegin() const noexcept;
    @\added{constexpr}@ const_iterator         cend() const noexcept;
    @\added{constexpr}@ const_reverse_iterator crbegin() const noexcept;
    @\added{constexpr}@ const_reverse_iterator crend() const noexcept;

    // 22.3.10.3, capacity
    [[nodiscard]] @\added{constexpr}@ bool empty() const noexcept;
    @\added{constexpr}@ size_type size() const noexcept;
    @\added{constexpr}@ size_type max_size() const noexcept;
    @\added{constexpr}@ void      resize(size_type sz);
    @\added{constexpr}@ void      resize(size_type sz, const T& c);

    // element access
    @\added{constexpr}@ reference       front();
    @\added{constexpr}@ const_reference front() const;
    @\added{constexpr}@ reference       back();
    @\added{constexpr}@ const_reference back() const;

    // 22.3.10.4, modifiers
    template<class... Args> @\added{constexpr}@ reference emplace_front(Args&&... args);
    template<class... Args> @\added{constexpr}@ reference emplace_back(Args&&... args);
    @\added{constexpr}@ void push_front(const T& x);
    @\added{constexpr}@ void push_front(T&& x);
    @\added{constexpr}@ void pop_front();
    @\added{constexpr}@ void push_back(const T& x);
    @\added{constexpr}@ void push_back(T&& x);
    @\added{constexpr}@ void pop_back();

    template<class... Args> @\added{constexpr}@ iterator emplace(const_iterator position, Args&&... args);
    @\added{constexpr}@ iterator insert(const_iterator position, const T& x);
    @\added{constexpr}@ iterator insert(const_iterator position, T&& x);
    @\added{constexpr}@ iterator insert(const_iterator position, size_type n, const T& x);
    template<class InputIterator>
    @\added{constexpr}@ iterator insert(const_iterator position, InputIterator first, InputIterator last);
    @\added{constexpr}@ iterator insert(const_iterator position, initializer_list<T> il);
    
    @\added{constexpr}@ iterator erase(const_iterator position);
    @\added{constexpr}@ iterator erase(const_iterator first, const_iterator last);
    @\added{constexpr}@ void     swap(list&)
      noexcept(allocator_traits<Allocator>::is_always_equal::value);
    @\added{constexpr}@ void     clear() noexcept;
    
    // 22.3.10.5, list operations
    @\added{constexpr}@ void splice(const_iterator position, list& x);
    @\added{constexpr}@ void splice(const_iterator position, list&& x);
    @\added{constexpr}@ void splice(const_iterator position, list& x, const_iterator i);
    @\added{constexpr}@ void splice(const_iterator position, list&& x, const_iterator i);
    @\added{constexpr}@ void splice(const_iterator position, list& x, const_iterator first, const_iterator last);
    @\added{constexpr}@ void splice(const_iterator position, list&& x, const_iterator first, const_iterator last);
    
    @\added{constexpr}@ size_type remove(const T& value);
    template<class Predicate> @\added{constexpr}@ size_type remove_if(Predicate pred);
    
    @\added{constexpr}@ size_type unique();
    template<class BinaryPredicate>
    @\added{constexpr}@ size_type unique(BinaryPredicate binary_pred);
    
    @\added{constexpr}@ void merge(list& x);
    @\added{constexpr}@ void merge(list&& x);
    template<class Compare> @\added{constexpr}@ void merge(list& x, Compare comp);
    template<class Compare> @\added{constexpr}@ void merge(list&& x, Compare comp);
    
    @\added{constexpr}@ void sort();
    template<class Compare> @\added{constexpr}@ void sort(Compare comp);
    
    @\added{constexpr}@ void reverse() noexcept;
  };

  template<class InputIterator,
           class Allocator = allocator<@\textit{iter-value-type<InputIterator>}@>>
    list(InputIterator, InputIterator, Allocator = Allocator())
      -> list<@\textit{iter-value-type<InputIterator>}@, Allocator>;

  // swap
  template<class T, class Allocator>
    @\added{constexpr}@ void swap(list<T, Allocator>& x, list<T, Allocator>& y)
      noexcept(noexcept(x.swap(y)));
}
\end{codeblock}%
\end{quote}

Change in \textbf{[list.cons] 22.3.10.2}:
\begin{quote}
\begin{itemdecl}
@\added{constexpr}@ explicit list(const Allocator&);
\end{itemdecl}
[...]
\begin{itemdecl}
@\added{constexpr}@ explicit list(size_type n, const Allocator& = Allocator());
\end{itemdecl}
[...]
\begin{itemdecl}
@\added{constexpr}@ list(size_type n, const T& value, const Allocator& = Allocator());
\end{itemdecl}
[...]
\begin{itemdecl}
template<class InputIterator>
  @\added{constexpr}@ list(InputIterator first, InputIterator last,
                             const Allocator& = Allocator());
\end{itemdecl}
[...]
\end{quote}

Change in \textbf{[list.capacity] 22.3.10.3}:
\begin{quote}
\begin{itemdecl}
@\added{constexpr}@ void resize(size_type sz);
\end{itemdecl}
[...]
\begin{itemdecl}
@\added{constexpr}@ void resize(size_type sz, const T& c);
\end{itemdecl}
[...]
\end{quote}

Change in \textbf{[list.modifiers] 22.3.10.4}:
\begin{quote}
\begin{itemdecl}
@\added{constexpr}@ iterator insert(const_iterator position, const T& x);
@\added{constexpr}@ iterator insert(const_iterator position, T&& x);
@\added{constexpr}@ iterator insert(const_iterator position, size_type n, const T& x);
template<class InputIterator>
  @\added{constexpr}@ iterator insert(const_iterator position, InputIterator first, InputIterator last);
@\added{constexpr}@ iterator insert(const_iterator position, initializer_list<T>);

template<class... Args> @\added{constexpr}@ reference emplace_front(Args&&... args);
template<class... Args> @\added{constexpr}@ reference emplace_back(Args&&... args);
template<class... Args> @\added{constexpr}@ iterator emplace(const_iterator position, Args&&... args);
@\added{constexpr}@ void push_front(const T& x);
@\added{constexpr}@ void push_front(T&& x);
@\added{constexpr}@ void push_back(const T& x);
@\added{constexpr}@ void push_back(T&& x);
\end{itemdecl}
[...]
\begin{itemdecl}
@\added{constexpr}@ iterator erase(const_iterator position);
@\added{constexpr}@ iterator erase(const_iterator first, const_iterator last);

@\added{constexpr}@ void pop_front();
@\added{constexpr}@ void pop_back();
@\added{constexpr}@ void clear() noexcept;
\end{itemdecl}
\end{quote}

Change in \textbf{[list.ops] 22.3.10.5}:
\begin{quote}
\begin{itemdecl}
@\added{constexpr}@ void splice(const_iterator position, list& x);
@\added{constexpr}@ void splice(const_iterator position, list&& x);

@\added{constexpr}@ void splice(const_iterator position, list& x, const_iterator i);
@\added{constexpr}@ void splice(const_iterator position, list&& x, const_iterator i);

@\added{constexpr}@ void splice(const_iterator position, list& x, const_iterator first,
const_iterator last);
@\added{constexpr}@ void splice(const_iterator position, list&& x, const_iterator first,
const_iterator last);
\end{itemdecl}
[...]
\begin{itemdecl}
@\added{constexpr}@ size_type remove(const T& value);
template<class Predicate> @\added{constexpr}@ size_type remove_if(Predicate pred);
\end{itemdecl}
[...]
\begin{itemdecl}
@\added{constexpr}@ size_type unique();
template<class BinaryPredicate> @\added{constexpr}@ size_type unique(BinaryPredicate binary_pred);
\end{itemdecl}
[...]
\begin{itemdecl}
@\added{constexpr}@ void merge(list& x);
@\added{constexpr}@ void merge(list&& x);
template<class Compare> @\added{constexpr}@ void merge(list& x, Compare comp);
template<class Compare> @\added{constexpr}@ void merge(list&& x, Compare comp);
\end{itemdecl}
[...]
\begin{itemdecl}
@\added{constexpr}@ void reverse() noexcept;
\end{itemdecl}
[...]
\begin{itemdecl}
@\added{constexpr}@ void sort();
template<class Compare> @\added{constexpr}@ void sort(Compare comp);
\end{itemdecl}
\end{quote}

Change in \textbf{[list.erasure] 22.3.10.6}:
\begin{quote}
\begin{itemdecl}
template<class T, class Allocator, class U>
  @\added{constexpr}@ void erase(list<T, Allocator>& c, const U& value);
  
template<class T, class Allocator, class Predicate>
  @\added{constexpr}@ void erase_if(list<T, Allocator>& c, Predicate pred);  
\end{itemdecl}
\end{quote}

\section{Implementation}
Possible implementation can be found here: \href{https://github.com/ZaMaZaN4iK/llvm-project/tree/feature/list_constexpr}{LLVM fork}. Notice that when proposal was written constexpr destructors were not supported in Clang. Also in this implementation isn't used \cc{operator<=>} - bunch of old operators used instead (just because libcxx at the moment doesn't use \cc{operator<=>} for \cc{std::forward_list}).

\section{References}
\renewcommand{\section}[2]{}%
\begin{thebibliography}{9}

\bibitem[P0784R1]{P0784R1}
  Multiple authors,
  \emph{Standard containers and constexpr}\newline
  \url{http://www.open-std.org/jtc1/sc22/wg21/docs/papers/2018/p0784r1.html}

\end{thebibliography}

\end{document}
