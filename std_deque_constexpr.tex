\documentclass{wg21}

\usepackage{xcolor}
\usepackage{soul}
\usepackage{ulem}
\usepackage{fullpage}
\usepackage{parskip}
\usepackage{csquotes}
\usepackage{listings}
\usepackage{minted}
\usepackage{enumitem}

\lstdefinestyle{base}{
  language=c++,
  breaklines=false,
  basicstyle=\ttfamily\color{black},
  moredelim=**[is][\color{green!50!black}]{@}{@},
  escapeinside={(*@}{@*)}
}

\newcommand{\cc}[1]{\mintinline{c++}{#1}}
\newminted[cpp]{c++}{}


\title{Making \cc{std::deque} constexpr}
\docnumber{P1923R0}
\audience{LEWGI}
\author{Alexander Zaitsev}{zamazan4ik@tut.by, zamazan4ik@gmail.com}

\begin{document}
\maketitle

\section{Revision history}
\begin{itemize}
  \item R0 -- Initial draft
\end{itemize}


\section{Abstract}
\cc{std::deque} is not currently \cc{constexpr} friendly. With the loosening
of requirements on \cc{constexpr} in \cite{P0784R1} and related papers, we
can now make \cc{std::deque} \cc{constexpr}, and we should in order to support
the \cc{constexpr} reflection effort (and other evident use cases).

\section{Motivation}
\cc{std::deque} is not so widely-used standard container as \cc{std::vector} or \cc{std::string}. But there is no reason to keep \cc{std::deque} in non-\cc{constexpr} state since one of the main directions of C++ evolution is compile-time programming. And we want to use in compile-time as much as possible from STL. And this paper makes \cc{std::deque} available in compile-time.

\section{Proposed wording}
We basically mark all the member and non-member functions of \cc{std::deque} \cc{constexpr}.

Direction to the editor: please apply \cc{constexpr} to all of \cc{std::deque},
including any additions that might be missing from this paper.

In \textbf{[support.limits.general]}, add the new feature test macro
\cc{__cpp_lib_constexpr_deque} with the corresponding value for header
\cc{<deque>} to Table 36 \textbf{[tab:support.ft]}.

Change in \textbf{[deque.syn] 22.3.3}:
\begin{quote}
\begin{codeblock}
#include <initializer_list>

namespace std {
  // 22.3.10, class template \tcode{deque}
  template<class T, class Allocator = allocator<T>> class deque;

  template<class T, class Allocator>
    @\added{constexpr}@ bool operator==(const deque<T, Allocator>& x, const deque<T, Allocator>& y);
  template<class T, class Allocator>
    @\added{constexpr}@ synth-three-way-result<T> operator<=>(const deque<T, Allocator>& x, const deque<T, Allocator>& y);

  template<class T, class Allocator>
    @\added{constexpr}@ void swap(deque<T, Allocator>& x, deque<T, Allocator>& y)
      noexcept(noexcept(x.swap(y)));
      
  template<class T, class Allocator, class U>
    @\added{constexpr}@ void erase(deque<T, Allocator>& c, const U& value);
  template<class T, class Allocator, class Predicate>
    @\added{constexpr}@ void erase_if(deque<T, Allocator>& c, Predicate pred);    

  [...]
}
\end{codeblock}
\end{quote}

Add after \textbf{[deque.overview] 22.3.8.1/2}:
\begin{quote}
\added{The types \texttt{iterator} and \texttt{const_iterator} meet the
constexpr iterator requirements ([iterator.requirements.general]).}
\end{quote}

Change in \textbf{[deque.overview] 22.3.8.1}:
\begin{quote}
\begin{codeblock}
namespace std {
  template<class T, class Allocator = allocator<T>>
  class deque {
  public:
    // types
    using value_type             = T;
    using allocator_type         = Allocator;
    using pointer                = typename allocator_traits<Allocator>::pointer;
    using const_pointer          = typename allocator_traits<Allocator>::const_pointer;
    using reference              = value_type&;
    using const_reference        = const value_type&;
    using size_type              = @\impdef@; // see 22.2
    using difference_type        = @\impdef@; // see 22.2
    using iterator               = @\impdef@; // see 22.2
    using const_iterator         = @\impdef@; // see 22.2
    using reverse_iterator       = std::reverse_iterator<iterator>;
    using const_reverse_iterator = std::reverse_iterator<const_iterator>;

    // 22.3.8.2, construct/copy/destroy
    @\added{constexpr}@ deque() : deque(Allocator()) { }
    @\added{constexpr}@ explicit deque(const Allocator&);
    @\added{constexpr}@ explicit deque(size_type n, const Allocator& = Allocator());
    @\added{constexpr}@ deque(size_type n, const T& value, const Allocator& = Allocator());
    template<class InputIterator>
      @\added{constexpr}@ deque(InputIterator first, InputIterator last, const Allocator& = Allocator());
    @\added{constexpr}@ deque(const deque& x);
    @\added{constexpr}@ deque(deque&& x);
    @\added{constexpr}@ deque(const deque& x, const Allocator&);
    @\added{constexpr}@ deque(deque&& x, const Allocator&);
    @\added{constexpr}@ deque(initializer_list<T>, const Allocator& = Allocator());
    
    @\added{constexpr}@ ~deque();
    @\added{constexpr}@ deque& operator=(const deque& x);
    @\added{constexpr}@ deque& operator=(deque&& x)
      noexcept(allocator_traits<Allocator>::is_always_equal::value);
    @\added{constexpr}@ deque& operator=(initializer_list<T>);
    template<class InputIterator>
      @\added{constexpr}@ void assign(InputIterator first, InputIterator last);
    @\added{constexpr}@ void assign(size_type n, const T& u);
    @\added{constexpr}@ void assign(initializer_list<T>);
    @\added{constexpr}@ allocator_type get_allocator() const noexcept;

    // iterators
    @\added{constexpr}@ iterator               begin() noexcept;
    @\added{constexpr}@ const_iterator         begin() const noexcept;
    @\added{constexpr}@ iterator               end() noexcept;
    @\added{constexpr}@ const_iterator         end() const noexcept;
    @\added{constexpr}@ reverse_iterator       rbegin() noexcept;
    @\added{constexpr}@ const_reverse_iterator rbegin() const noexcept;
    @\added{constexpr}@ reverse_iterator       rend() noexcept;
    @\added{constexpr}@ const_reverse_iterator rend() const noexcept;

    @\added{constexpr}@ const_iterator         cbegin() const noexcept;
    @\added{constexpr}@ const_iterator         cend() const noexcept;
    @\added{constexpr}@ const_reverse_iterator crbegin() const noexcept;
    @\added{constexpr}@ const_reverse_iterator crend() const noexcept;

    // 22.3.8.3, capacity
    [[nodiscard]] @\added{constexpr}@ bool empty() const noexcept;
    @\added{constexpr}@ size_type size() const noexcept;
    @\added{constexpr}@ size_type max_size() const noexcept;
    @\added{constexpr}@ void      resize(size_type sz);
    @\added{constexpr}@ void      resize(size_type sz, const T& c);
    @\added{constexpr}@ void      shrink_to_fit();

    // element access
    @\added{constexpr}@ reference       operator[](size_type n);
    @\added{constexpr}@ const_reference operator[](size_type n) const;
    @\added{constexpr}@ reference       at(size_type n);
    @\added{constexpr}@ const_reference at(size_type n) const;
    @\added{constexpr}@ reference       front();
    @\added{constexpr}@ const_reference front() const;
    @\added{constexpr}@ reference       back();
    @\added{constexpr}@ const_reference back() const;

    // 22.3.8.4, modifiers
    template<class... Args> @\added{constexpr}@ reference emplace_front(Args&&... args);
    template<class... Args> @\added{constexpr}@ reference emplace_back(Args&&... args);
    template<class... Args> @\added{constexpr}@ iterator emplace(const_iterator position, Args&&... args);
    
    @\added{constexpr}@ void push_front(const T& x);
    @\added{constexpr}@ void push_front(T&& x);
    @\added{constexpr}@ void push_back(const T& x);
    @\added{constexpr}@ void push_back(T&& x);
    
    @\added{constexpr}@ iterator insert(const_iterator position, const T& x);
    @\added{constexpr}@ iterator insert(const_iterator position, T&& x);
    @\added{constexpr}@ iterator insert(const_iterator position, size_type n, const T& x);
    template<class InputIterator>
      @\added{constexpr}@ iterator insert(const_iterator position, InputIterator first, InputIterator last);
    @\added{constexpr}@ iterator insert(const_iterator position, initializer_list<T>);
    
    @\added{constexpr}@ void pop_front();
    @\added{constexpr}@ void pop_back();
    
    @\added{constexpr}@ iterator erase(const_iterator position);
    @\added{constexpr}@ iterator erase(const_iterator first, const_iterator last);
    @\added{constexpr}@ void     swap(deque&)
      noexcept(allocator_traits<Allocator>::is_always_equal::value);
    @\added{constexpr}@ void     clear() noexcept;
  };

  template<class InputIterator,
           class Allocator = allocator<@\textit{iter-value-type<InputIterator>}@>>
    deque(InputIterator, InputIterator, Allocator = Allocator())
      -> deque<@\textit{iter-value-type<InputIterator>}@, Allocator>;

  // swap
  template<class T, class Allocator>
    @\added{constexpr}@ void swap(deque<T, Allocator>& x, deque<T, Allocator>& y)
      noexcept(noexcept(x.swap(y)));
}
\end{codeblock}%
\end{quote}

Change in \textbf{[deque.cons] 22.3.8.2}:
\begin{quote}
\begin{itemdecl}
@\added{constexpr}@ explicit deque(const Allocator&);
\end{itemdecl}
[...]
\begin{itemdecl}
@\added{constexpr}@ explicit deque(size_type n, const Allocator& = Allocator());
\end{itemdecl}
[...]
\begin{itemdecl}
@\added{constexpr}@ deque(size_type n, const T& value, const Allocator& = Allocator());
\end{itemdecl}
[...]
\begin{itemdecl}
template<class InputIterator>
  @\added{constexpr}@ deque(InputIterator first, InputIterator last,
                             const Allocator& = Allocator());
\end{itemdecl}
[...]
\end{quote}

Change in \textbf{[deque.capacity] 22.3.8.3}:
\begin{quote}
\begin{itemdecl}
@\added{constexpr}@ void resize(size_type sz);
\end{itemdecl}
[...]
\begin{itemdecl}
@\added{constexpr}@ void resize(size_type sz, const T& c);
\end{itemdecl}
[...]
\begin{itemdecl}
@\added{constexpr}@ void shrink_to_fit();
\end{itemdecl}
[...]
\end{quote}

Change in \textbf{[deque.modifiers] 22.3.8.4}:
\begin{quote}
\begin{itemdecl}
@\added{constexpr}@ iterator insert(const_iterator position, const T& x);
@\added{constexpr}@ iterator insert(const_iterator position, T&& x);
@\added{constexpr}@ iterator insert(const_iterator position, size_type n, const T& x);
template<class InputIterator>
  @\added{constexpr}@ iterator insert(const_iterator position,
                                      InputIterator first, InputIterator last);
@\added{constexpr}@ iterator insert(const_iterator position, initializer_list<T>);

template<class... Args> @\added{constexpr}@ reference emplace_front(Args&&... args);
template<class... Args> @\added{constexpr}@ reference emplace_back(Args&&... args);
template<class... Args> @\added{constexpr}@ iterator emplace(const_iterator position, Args&&... args);
@\added{constexpr}@ void push_front(const T& x);
@\added{constexpr}@ void push_front(T&& x);
@\added{constexpr}@ void push_back(const T& x);
@\added{constexpr}@ void push_back(T&& x);
\end{itemdecl}
[...]
\begin{itemdecl}
@\added{constexpr}@ iterator erase(const_iterator position);
@\added{constexpr}@ iterator erase(const_iterator first, const_iterator last);
@\added{constexpr}@ void pop_front();
@\added{constexpr}@ void pop_back();
\end{itemdecl}
[...]
\end{quote}

Change in \textbf{[deque.erasure] 22.3.8.5}:
\begin{quote}
\begin{itemdecl}
template<class T, class Allocator, class U>
  @\added{constexpr}@ void erase(deque<T, Allocator>& c, const U& value);
  
template<class T, class Allocator, class Predicate>
  @\added{constexpr}@ void erase_if(deque<T, Allocator>& c, Predicate pred);  
\end{itemdecl}
\end{quote}

\section{Implementation}
Possible implementation can be found here: \href{https://github.com/ZaMaZaN4iK/llvm-project/tree/feature/deque_constexpr}{LLVM fork}. Notice that when proposal was written constexpr destructors were not supported in Clang. Also in this implementation isn't used \cc{operator<=>} - bunch of old operators used instead (just because libcxx at the moment doesn't use \cc{operator<=>} for \cc{std::deque}).

\section{References}
\renewcommand{\section}[2]{}%
\begin{thebibliography}{9}

\bibitem[P0784R1]{P0784R1}
  Multiple authors,
  \emph{Standard containers and constexpr}\newline
  \url{http://www.open-std.org/jtc1/sc22/wg21/docs/papers/2018/p0784r1.html}

\end{thebibliography}

\end{document}
