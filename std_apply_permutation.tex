\documentclass{wg21}

\usepackage{xcolor}
\usepackage{soul}
\usepackage{ulem}
\usepackage{fullpage}
\usepackage{parskip}
\usepackage{csquotes}
\usepackage{listings}
\usepackage{minted}
\usepackage{enumitem}

\lstdefinestyle{base}{
  language=c++,
  breaklines=false,
  basicstyle=\ttfamily\color{black},
  moredelim=**[is][\color{green!50!black}]{@}{@},
  escapeinside={(*@}{@*)}
}

\newcommand{\cc}[1]{\mintinline{c++}{#1}}
\newminted[cpp]{c++}{}


\title{Add \cc{std::apply_permutation} algorithm}
\docnumber{P0000}
\audience{LEWGI}
\author{Alexander Zaitsev}{zamazan4ik@tut.by, zamazan4ik@gmail.com}

\begin{document}
\maketitle

\section{Revision history}
\begin{itemize}
  \item R0 -- Initial draft
\end{itemize}

\section{Issues}
\begin{itemize}
    \item \cc{apply_reverse_permutation} name is a little bit ugly. Suggestions are welcomed.
\end{itemize}

\section{Motivation}

\section{Proposed wording}
Add to \textbf{[alg.modifying.operations] 25.6} new subsection \textbf{[alg.permute] Permute 25.6.15}
after \textbf{[alg.shift] 25.6.14}:
\begin{quote}
\begin{itemdecl}
template<class RandomAccessIterator1, class RandomAccessIterator2>
  constexpr void apply_permutation(RandomAccessIterator1 item_first,
                                   RandomAccessIterator1 item_last,
                                   RandomAccessIterator2 ind_first,
                                   RandomAccessIterator2 ind_last);
template<class ExecutionPolicy, class RandomAccessIterator1, class RandomAccessIterator2>
  constexpr void apply_permutation(ExecutionPolicy&& exec,
                                   RandomAccessIterator1 item_first,
                                   RandomAccessIterator1 item_last,
                                   RandomAccessIterator2 ind_first,
                                   RandomAccessIterator2 ind_last);

template<random_access_iterator I1, random_access_iterator I2, sentinel_for<I1> S1,
         sentinel_for<I2> S2, class Proj = identity>
  constexpr I1 ranges::apply_permutation(I1 item_first, S1 item_last, I2 ind_first, S2 ind_last,
                                         Proj proj = {});
template<random_access_range R1, random_access_range R2, class Proj = identity>
  constexpr safe_iterator_t<R1> ranges::apply_permutation(R1&& r1, R2&& r2, Proj proj = {});                                         
\end{itemdecl}

    Let proj be identity{} for the overloads with no parameter named proj.

    Effects: Reorders the elements in the range [item_first, item_last) according to the order sequence [ind_first, ind_last). Every value in order sequence means where the item comes from.
    
    Requires: For the overloads in namespace std, RandomAccessIterator1 shall meet the Cpp17ValueSwappable requirements. Order sequence needs to be exactly a permutation of the sequence [0, 1, ... , N], where N is the biggest index in the item sequence (zero-indexed).

    Returns: item_last, for the overloads in namespace ranges.

    Complexity: Linear.
    
    Note: Order sequense gets permuted.

\begin{itemdecl}
template<class RandomAccessIterator1, class RandomAccessIterator2>
  constexpr void apply_reverse_permutation(RandomAccessIterator1 item_first,
                                           RandomAccessIterator1 item_last,
                                           RandomAccessIterator2 ind_first,
                                           RandomAccessIterator2 ind_last);
template<class ExecutionPolicy, class RandomAccessIterator1, class RandomAccessIterator2>
  constexpr void apply_permutation(ExecutionPolicy&& exec,
                                   RandomAccessIterator1 item_first,
                                   RandomAccessIterator1 item_last,
                                   RandomAccessIterator2 ind_first,
                                   RandomAccessIterator2 ind_last);
                                               
template<random_access_iterator I1, random_access_iterator I2,
         sentinel_for<I1> S1, sentinel_for<I2> S2, class Proj = identity>
  constexpr I1 ranges::apply_reverse_permutation(I1 item_first, S1 item_last,
                                                 I2 ind_first, S2 ind_last, Proj proj = {});                                           	
template<random_access_range R1, random_access_range R2, class Proj = identity>
  constexpr safe_iterator_t<R1> ranges::apply_reverse_permutation(R1&& r1, R2&& r2,
                                                                  Proj proj = {});  
\end{itemdecl}

    Let proj be identity{} for the overloads with no parameter named proj.
    
    Effects: Reorders the elements in the range [item_first, item_last) according to the order sequence [ind_first, ind_last). Every value in order sequence means where the item goes to.
    
    Requires: For the overloads in namespace std, RandomAccessIterator1 shall meet the Cpp17ValueSwappable requirements. Order sequence needs to be exactly a permutation of the sequence [0, 1, ... , N], where N is the biggest index in the item sequence (zero-indexed).
    
    Returns: item_last, for the overloads in namespace ranges.
    
    Complexity: Linear.
    
    Note: Order sequense gets permuted.
\end{quote}

\section{Examples}
Given the containers: \cc{std::vector<int> emp_vec, emp_order, one{1}, one_order{0}, two{1,2}},
\cc{two_order{1,0}, vec{1, 2, 3, 4, 5}, order{4, 2, 3, 1, 0}}, then 
\begin{quote}
\begin{itemdecl}    
apply_permutation(emp_vec, emp_order))  // no changes
apply_reverse_permutation(emp_vec, emp_order))  // no changes
apply_permutation(one, one_order)  // no changes
apply_reverse_permutation(one, one_order)  // no changes
apply_permutation(two, two_order)  // two:{2,1}
apply_reverse_permutation(two, two_order)  // two:{2,1}
apply_permutation(vec, order)  // vec:{5, 3, 4, 2, 1}
apply_reverse_permutation(vec, order)  // vec:{5, 4, 2, 3, 1}
\end{itemdecl}
\end{quote}


\section{Possible implementation}
Possible implementation (without version with Execution policy) can be found in Boost.Algorithm: \href{https://github.com/boostorg/algorithm/blob/master/include/boost/algorithm/apply_permutation.hpp}{GitHub}. Documentation can be found here: \href{https://www.boost.org/doc/libs/1_70_0/libs/algorithm/doc/html/the_boost_algorithm_library/Misc/apply_permutation.html}{Boost}. Available in Boost.Algorithm since Boost 1.68.
Original implementation and ideas from Microsoft blog: \href{https://blogs.msdn.microsoft.com/oldnewthing/20170102-00/?p=95095}{1}, \href{https://blogs.msdn.microsoft.com/oldnewthing/20170103-00/?p=95105}{2}, \href{https://blogs.msdn.microsoft.com/oldnewthing/20170104-00/?p=95115}{3}, \href{https://blogs.msdn.microsoft.com/oldnewthing/20170109-00/?p=95145}{4}, \href{https://blogs.msdn.microsoft.com/oldnewthing/20170110-00/?p=95155}{5}, \href{https://blogs.msdn.microsoft.com/oldnewthing/20170111-00/?p=95165}{6}.
\end{document}
